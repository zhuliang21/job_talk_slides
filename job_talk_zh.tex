\documentclass[professionalfonts, aspectratio=169]{beamer}  
\usepackage{CJKutf8}
\AtBeginDocument{\begin{CJK}{UTF8}{gkai}}
\AtEndDocument{\end{CJK}}
\usefonttheme{serif}                                        % Font theme: serif
\usepackage[utf8]{inputenc}                                 % Allows the use of different input encodings
\usepackage{graphicx}                                       % Enhanced support for graphics
\usepackage{amsmath}                                        % Enhances the typesetting of mathematics
\usepackage{amsfonts}                                       % Extra mathematical fonts
\usepackage{amssymb}                                        % Extra mathematical symbols
\usepackage{threeparttable}                 % For table notes
\usepackage{tikz}
\usetikzlibrary{shapes,arrows,positioning}
\usepackage{booktabs}                                       % Enhances the quality of tables
\usepackage{pgfplots}
\pgfplotsset{compat=1.17}
\usepackage{multirow}
% link settings
\usepackage{hyperref}                       % For hyperlinks
\usepackage[authoryear]{natbib}             % For bibliography

\usepackage{appendixnumberbeamer} % Numbering appendixes in beamer
\pdfstringdefDisableCommands{
  \def\translate{}
}
\usepackage{dcolumn}
\setbeamertemplate{navigation symbols}{} % Remove navigation symbols
\setbeamertemplate{footline}[frame number] % Add page number

% \setbeamercovered{transparent}  % Semi-transparent overlay

%---- Set the title page ----%
\title{保险市场中的风险调整、自选择与计划设计}
\institute{Stony Brook University}
\author{梁祝}
\date{\today}

%---- Begin the document ----%
\begin{document}

\AtBeginSection[] % At the beginning of each section...
{
  \begin{frame}[noframenumbering, plain] % Create a new frame without a frame number
    \tableofcontents[currentsection] % Display the table of contents for the current section
  \end{frame}
}

\begin{frame}[noframenumbering, plain] % title frame without a frame number
    \titlepage
\end{frame}

%---- The slides structure ----%

\begin{frame}{管理型竞争}
  \begin{itemize}
      \item 政府向保险公司提供固定且预先确定的受益人补贴(定额支付),不依据实际医疗支出。
      \item 保险公司为受益人提供保险计划,并向服务提供者支付部分实际医疗费用。
      \item 保险公司不得进行价格歧视。
  \end{itemize}
\end{frame}

\begin{frame}{自选择}
  \textbf{消费者的自选择}:
  \begin{itemize}
    \item 相较于可观察指标,消费者对自身健康状况有更深入的了解。
    \item 基于这些私有信息,选择最适合的保险计划。
    \item 健康状况良好的消费者更倾向于选择提供较低慷慨度的保险计划。
  \end{itemize}
\end{frame}

\begin{frame}{客户筛选}
  \textbf{保险公司的策略性设计}:
  \begin{itemize}
    \item 健康消费者的医疗支出较低,利润率更高,激励保险公司筛选这些消费者。
    \item 政府通过基于可观察的消费者特征进行风险调整定额支付,以消除这种激励。
    \item 现有风险调整未能涵盖私有信息,无法全面反映风险,\\激励因此未完全消除。
      \hyperlink{simplifiedRiskAdjustment}{\beamerbutton{简化的风险调整例子}}
    \item 保险公司利用自选择效应设计保险计划,从而完成筛选。
  \end{itemize}
\end{frame}

\begin{frame}{研究问题}
  \textbf{管理型竞争下的保险市场}:
  \begin{itemize}
    \item 消费者的自选择如何影响保险计划的策略性设计?
    \item 如何评估自选择效应对保险市场福利的影响?
  \end{itemize}
\end{frame}

\begin{frame}{研究方法}
  \begin{itemize}
    \item 使用美国医疗保险市场Medicare Advantage的数据,包括消费者层面和保险计划层面的信息。
    \item 建立结构模型,涵盖消费者(需求端)和保险公司(供给端)的行为。
    \item 需求端模型考虑消费者的私有信息及自选择效应。
    \item 供给端模型涉及定价选择和保险计划慷慨度的设计。
    \item 进行反事实模拟,分析在消除自选择效应下保险计划设计及福利的变化。
  \end{itemize}
\end{frame}

\begin{frame}{关键结论}
  \begin{itemize}
    \item 私有信息显著影响消费者的保险计划选择。
    \item 当前的风险调整机制未能完全弥补选择效应带来的激励。
    \item 消费者的自选择激励保险公司设计提供较低慷慨度的保险计划。
    \item 反事实模拟表明,如果风险调整机制完全消除选择效应,保险计划的慷慨度将显著提高,消费者福利提升11\%,而保险公司的利润提升34.6\%。
  \end{itemize}
\end{frame}

\begin{frame}{贡献}
  \begin{itemize}
    \item \textbf{理论}: 发展了一个管理型竞争模型,允许消费者自选择,并内生化了保险计划慷慨度设计。
    \item \textbf{实证}: 将模型应用于Medicare Advantage市场,从风险调整的视角进行切入,评估自选择效应对福利的影响。
    \item \textbf{政策}: 为改进风险调整支付政策提供量化分析。
  \end{itemize}
\end{frame}

\begin{frame}
  \centering
  \Huge 感谢您的时间!
\end{frame}


%---- The appendix ----%
\appendix
\begin{frame}[plain, noframenumbering] % Create a new frame without a frame number
  % show the title of the appendix centered on the slide
  \begin{center}
    \Huge 附录
  \end{center}
\end{frame}

\begin{frame}{简化的风险调整例子}\label{simplifiedRiskAdjustment}
  \begin{itemize}
    \item 年轻和年老个体数量相等
    \begin{itemize}
      \item \textbf{年轻}: 健康的占80\%,生病的占20\%
      \item \textbf{年老}: 健康的占20\%,生病的占80\%
    \end{itemize}
    \item 医疗费用:健康个体 \$1,000,生病个体 \$5,000
    \item 年龄对政府是可观察到的;健康状况则不可 (私有信息) \pause
    \item 按年龄进行风险调整的定额支付:
    \begin{itemize}
      \item \textbf{年轻}: \$1,000 $\times$ 0.8 + \$5,000 $\times$ 0.2 = \$1,800
      \item \textbf{年老}: \$1,000 $\times$ 0.2 + \$5,000 $\times$ 0.8 = \$4,200
    \end{itemize}
    \item 基于健康状况的平均定额支付:
    \begin{itemize}
      \item \textbf{健康}: \$1,800 $\times$ 0.8 + \$4,200 $\times$ 0.2 = \$2,040\ (\textbf{高于}费用 \$1,000)
      \item \textbf{生病}: \$1,800 $\times$ 0.2 + \$4,200 $\times$ 0.8 = \$3,960\ (\textbf{低于}费用 \$5,000)
    \end{itemize}
    \item 公司 \textbf{仍然偏好}健康个体。
  \end{itemize}
\end{frame}


\end{document}